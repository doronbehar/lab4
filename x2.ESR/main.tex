\documentclass{article}

\usepackage{tabularx}

% https://tex.stackexchange.com/a/129100/125609, to draw the pico ampremeter
\usepackage{tikz}
\usetikzlibrary{arrows}
\usetikzlibrary{shapes}
\newcommand*\circled[1]{\tikz[baseline=(char.base)]{
		            \node[shape=circle,draw,inner sep=1pt] (char) {#1};}}

% NOTE: To put equations in their environment we need either `float` or
% `caption`.  We use float to put equations and other environments exactly
% where they appear in the code with the `H` placeholder, and for that we
% redefine the `equ` environment sort of twice, so this is a bit flaky but
% it works.
\usepackage{caption}
\usepackage{subcaption}

\DeclareCaptionType{equ}[][]
\captionsetup[equ]{name=נוסחא}
\usepackage{float}
\floatstyle{plain}
% https://www.overleaf.com/learn/latex/Positioning_of_Figures
\newfloat{equ}{H}{eq}[section]
\floatname{equ}{נוסחא}

\DeclareCaptionType{graph}[][]
\captionsetup[graph]{name=גרף }

% to includegraphics
\usepackage{graphicx}

% to fix itemize lists:
% https://tex.stackexchange.com/a/53453/125609
\usepackage{enumitem}
\setlist[itemize,1]{label={\fontfamily{cmr}\fontencoding{T1}\selectfont\textbullet}}

% To crop inserted images: https://tex.stackexchange.com/questions/57418/crop-an-inserted-image
\usepackage[export]{adjustbox}

% Links
\usepackage{hyperref}
\hypersetup{colorlinks = true,
	citecolor = gray,
	linkcolor = red,
	citecolor = green,
	filecolor = magenta,
	urlcolor = cyan
}

\usepackage[version=4]{mhchem}

% To include plots by matplotlib
\usepackage{pgf}
\usepackage{pgfplots}
\pgfplotsset{compat=newest}
% Note we use resizebox as explained here through out the document https://tex.stackexchange.com/a/582956/125609

\usepackage{geometry}
 \geometry{
 a4paper,
 top=30mm,
 left = 25mm,
 right = 25mm,
 bottom=30mm,
 headheight=2cm,
 headsep=2cm,
 footskip=1.5cm
}

% Language
\usepackage{polyglossia}
\setdefaultlanguage{hebrew}
\setotherlanguage{english}
\usepackage{hebrewcal}

\usepackage[
backend=biber,
isbn=false,
style=numeric,
doi = false,
sorting=ynt
]{biblatex}
% Seems to be a recommended package but it makes quotes in bibliography at the
% end appear with a question mark instead of `"`.
%\usepackage{csquotes}
\addbibresource{references.bib} % Imports bibliography file

% Fonts
\setmainfont{David CLM}
\setsansfont{Libertinus Serif}
\setmonofont{FreeMono}
\newfontfamily\hebrewfont{David CLM}[Script=Hebrew]
\newfontfamily\hebrewfontsf{Libertinus Serif}[Script=Hebrew]
\newfontfamily\hebrewfonttt{FreeMono}[Script=Hebrew]

\title{

} 
\author{
שרה לחצר ודורון בכר \\
הפקולטה לפיזיקה, טכניון - מכון טכנולוגי לישראל.
}
\date{\today}

\begin{document}
\maketitle

\begin{abstract}



\end{abstract}

\section{מבוא}
בניסוי זה השתמשנו בשיטת סחריר האלקטרון, שיטה בה חוקרים תכונות של חומרים בעלי אלקטרונים לא מזווגים.
בתהודת סחריר האלקטרון 
\textenglish{ESR},
נמדדת בליעת אנרגיה של אלקטרונים בנוכחות שדה מגנטי חיצוני. השדה המגנטי גורם לאלקטרונים להתייצב במקביל ובאנטי מקביל ביחס אליו, כתלות במומנטים המגנטיים של האלקטרונים וליצור מצבי אנרגיה חדשים.
מתוך אפקט זימן נוכל לקבוע כי הפרשי האנרגיות נתונים בנוסחא
\label{equ:U_gap}.

\begin{equ}
$$ \Delta U = g \mu _B H$$
\caption{
פער אנרגטי בין רמות האנרגיה שנוצרו משדה מגנטי חיצוני-
$H$,
המגנטון של בוהר-
$\mu _B$,
וקבוע הפיצול
$g$.
}

פילוג האלקטרונים בשתי רמות האנרגיה יהיה בהתאם להתפלגות בולצמן, כך שהרמה האנרגטית הנמוכה תהיה מאוכלסת יותר מהרמה האנרגית הגבוהה .

% TODO: read about stimulated emission and explain why there is an equal probability for transitions, maybe explain with fermi golden rule why most of the transitions will be from - to +.
לפיכך, בהכנסת פוטונים בעלי אנרגיה התואמת את נוסחא 
\label{equ:U_gap}
, יעברו יותר אלקטרונים מהרמה התחתונה לעליונה מאשר להיפך, כך שתבלע אנרגיה במערכת.
\label{equ:U_gap}
\end{equ}


\section{מערכת הניסוי}


בניגוד למקובל בספקטרוסקופיה בה משנים את אספקת האנרגיה לדגם, בניסוי זה שינינו את השדה המגנטי החיצוני וכך גם את רמות האנרגיה.


\end{document}